\section{Introduction}
The goal of this project is to realize a smart-home application to control the temperature of different room in order to minimize the consumption of the entire building.

The system is composed by two differ modules

\subsection{Room module}
The aim of this module is to periodically get the temperature and the motion inside a room,
act on the valve in order to achieve a goal temperature and send the status of the room to the
Central unit.

The module is composed by:
\begin{itemize}
	\item Temperature sensor
	\item Humidity sensor
	\item Motion sensor
	\item Valve actuator
	\item Wireless communication module
\end{itemize}

\subsection{Temperature control}
The module act on a servomotor that control a valve in order to adjust the temperature of the room.
The valve is set to different positions based on the temperature error (difference between the actual temperature and the desired themperature):
\begin{itemize}
	\item if the temperature error is below then -COLD_VALUE the valve is moved in OPEN_POSITION
	\item if the temperature error is below then -WARM_VALUE the valve is moved in 3/4
	\item if the temperature error is between a [-APPROCHING, +APPROCHING] the valve is moved in HALF_POSITION
	\item if the temperature error is greater then WARM_VALUE the valve is moved in 1/4 position
	\item if the temperature error is greater then HOT_VALUE the valve is moved in CLOSED_POSITION
\end{itemize}


\subsection{Central unit}