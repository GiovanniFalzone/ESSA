\section{Room Module}
The aim of this module is to periodically get the temperature and the motion inside a room,
act on the valve in order to achieve a goal temperature and send the status of the room to the
Central unit.

The module is composed by:
\begin{itemize}
	\item Temperature sensor
	\item Humidity sensor
	\item Motion sensor
	\item Valve actuator
	\item Wireless communication module
\end{itemize}

\subsection{Temperature control}
The module act on a servomotor that control a valve in order to adjust the temperature of the room.
The valve is set to different positions based on the temperature error (difference between the actual temperature and the desired themperature):
\begin{itemize}
	\item if the temperature error is below then -COLD\_VALUE the valve is moved in OPEN\_POSITION
	\item if the temperature error is below then -WARM\_VALUE the valve is moved in 3/4
	\item if the temperature error is between a [-APPROCHING, APPROCHING] the valve is moved in HALF\_POSITION
	\item if the temperature error is greater then WARM\_VALUE the valve is moved in 1/4 position
	\item if the temperature error is greater then HOT\_VALUE the valve is moved in CLOSED\_POSITION
\end{itemize}

\subsection{Energy Saving mode}
In order to minimize the consumpion the module keep tracks on the precense of people inside the room
using a motion sensor.
If a motion is detected the module know that and set the desired temperature room temperature to the one set by the user.
If there is the number of motions detected by the sensor is not sufficient to say that someone is inside the room then the module set
the desired temperature to a value equal to the user desired temperature minus a default value (-1\degree).

\newpage
\subsection{Functional requirements}

\newpage{Hardware implementation}