\section{Introduction}
The purpose of this project is to realize a smart-home application to control the heating system of a building based on the temperature of each room, in order to minimize the consumption of the building each room apply an energy saving function reducing the desired temperature when it is not needed.
The system is composed by two differ modules
\begin{itemize}
	\item central unit module 
	\item room module
\end{itemize}

\subsection{Central Unit}
The \textit{Central Unit} has the role of coordinator that retrieve the status of each room and compute the average values for the builng.\\

\subsubsection{Graphical user interface}
Using a graphical User Interface the module represents the average values of the building and the values for each room, the graphical User Interface is composed by:
\begin{itemize}
	\item \textit{Main page} to represent the overview of the building
	\item \textit{Room page} to represent the status of each room
	\item \textit{Setting page} to set the desired temperature
\end{itemize}
In the \textit{Settings page} the module shall allow the user to set the desired temperature shared with the rooms.
The minimum and maximum allowed temperatures are 15 C\degree and 30 C\degree.

When the \textit{Main page} is selected the module shall represent the average values among all the rooms for \textit{Temperature}, \textit{Humidity} and \textit{Usage}.
When the \textit{Main page} is selected the module shall represent the \textit{Energy Saving} if at least one room is set to \textit{Energy Saving mode}.
When the \textit{Main page} is selected the module shall represent the \textit{Warning} if at least one room is set to \textit{crashed}.

The graphical user interface shall represent the following informations as reported in the following table \ref{tab:GraphicalInformations}.
\begin{table}[H]
	\centering
			\begin{tabular}{||l | l||} 
			\hline
			\textbf{Information}	& \textbf{Format} \\ 
			\hline
			Temperature	& C\degree \\ 
			\hline
			Humidity	& \% \\ 
			\hline
			Usage		& \% \\ 
			\hline
			Energy Saving	& boolean \\ 
			\hline
			Warning		& boolean \\ 
			\hline
		\end{tabular}
		\captionof{table}{\label{tab:GraphicalInformations}}
\end{table}

\subsubsection{Communication}
Whenever a \textit{RoomRequest Message} is sent and the \textit{RoomStatus Message} is not received within 20s the module shall mark the room as \textbf{crashed}.
The module shall send the \textit{RoomRequest Message} for each room at least every 30s.

%--------------------------------------
\subsection{Room Module}
The purpose of this module is to control the temperature of the room acting on a valve in order to 
reach and mantain the \textit{GoalTemperature}.

\subsubsection{Energy Saving mode}
In order to minimize the consumpion the module keep tracks of the presence of motion inside the room
using a motion sensor. \\

If a motion is detected in the last 30s, the module shall set the \textit{GoalTemperature} to the one set by the user,otherwise the module shall set the \textit{GoalTemperature} to:

\begin{equation}
	GoalTemperature = DesiredTemperature - EnergySavingTemperatureOffset
\end{equation}
Whenever the module is in \textit{Energy Saving mode} it shall notify through the \textit{Interface}.

\subsubsection{Valve control}
In order to control the heating of the room the valve is moved to different positions based on the temperature error 
(\textit{ActualTemperature} - \textit{GoalThemperature}) as in the following table, 
whenever one of these rules is valid the module shall move the valve in the correspondant position described in the third column of the table \ref{tab:ValvePositions} as percentage of maximum flow.
\begin{center}
	\begin{tabular}{| l | l | l |} 
		\hline
		\textbf{rule} & \textbf{valve position} & \textbf{Flow in \%}\\
		\hline
		\begin{math} error < -HIGH \end{math} &  OPEN\_POSITION & 100\\
		\hline
		\begin{math} error \in [-HIGH, -APPROCHING) \end{math}  & HIGH\_POSITION & 75\\
		\hline
		\begin{math} error \in [-APPROCHING, +APPROCHING] \end{math} & MIDDLE\_POSITION & 50 \\
		\hline
		\begin{math} error \in (APPROCHING, HIGH] \end{math} & LOW\_POSITION & 25\\
		\hline
		error > HIGH &  CLOSED\_POSITION & 0 \\
		\hline
	\end{tabular}
	\captionof{table}{\label{tab:ValvePositions}}
\end{center}
Whenever the valve is in \textit{OPEN\_POSITION} or \textit{CLOSED\_POSITION} the module shall check the position and set \textbf{Valve Error} if the position is not valid.

In the following table\ref{tab:TemperatureThresholds} are reported the temperature thresholds:
\begin{center}
	\begin{tabular}{||l | l||} 
		\hline
		HIGH 		& 2 C\degree \\ 
		\hline
		APPROCHING 	& 1 C\degree \\ 
		\hline
	\end{tabular}
	\captionof{table}{\label{tab:TemperatureThresholds}}
\end{center}


\subsubsection{Communication}
Whenever a \textit{RoomRequest Message} is not received within 60s the module shall send the \textit{RoomStatus Message} and set the \textbf{Communication Error}.

\subsubsection{Errors}
Whenever one of the errors are set, the module shall notify it through the \textit{Interface}.
In the following table \ref{tab:RoomErrors} are reported the possible errors.
\begin{center}
	\begin{tabular}{|| l ||} 
		\hline
		\textbf{Valve Error} \\ 
		\hline
		\textbf{Communication Error} \\ 
		\hline
		\textbf{Sensor Error} \\ 
		\hline
	\end{tabular}
	\captionof{table}{\label{tab:RoomErrors}}
\end{center}
